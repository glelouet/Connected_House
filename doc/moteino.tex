\documentclass{report}
\usepackage[french]{babel}
\usepackage[utf8]{inputenc}
\usepackage{hyperref}

\title{Utilisation du projet Moteino}

\begin{document}

\chapter{Contexte}

Le Moteino est un ordinateur de très petite taille et peu cher. Il est basé sur l'Arduino, très modulaire. 

Le but de ce projet est de prendre en main l'architecture embarquée du Moteino : produire du code C++ à installer sur un Moteino, mais surtout comprendre comment manipuler correctement cette architecture.
La production de code C++ demande l'utilisation d'outils particuliers. Arduino propose un IDE minimaliste qui ne convient pas aux projets industriels, nous utilisons donc platformio pour automatiser la gestion des librairies, compiler et envoyer le code sur les chipsets.

\chapter{Liens utiles}

\href{https://github.com/glelouet/Moteino}{le git du projet}

\href{http://platformio.org/}{Le site de platformio}

\chapter{Organisation du projet}

\section{Code source}

Le répetoire src/ contient trois répertoires:
\begin{description}
\item[src/main] contient les sources principaux à installer sur les chipsets. Chaque répertoire contient un module à installer sur un Moteino.
\item[src/lib] contient les librairies que nous développons, utilisées dans les main.
\item[src/ext] contient le librairies que nous avons importées, aprè d'éventuelles modifications, depuis d'autres sites.
\item[src/test] Contient les tests (pas utilisé pour l'instant)
\end{description}

\section{Documentation}

Le répertoire doc/ contient la documentation de ce projet, en particulier ce fichier.

\section{Outils de développement}

\subsection{Gestionnaire de version : git}

\subsection{Interface graphique : Atom/Platformio}

J'utilise pio pour le déploiement en ligne de commande, ça me parraissait plus pratique d'utiliser l'IDE qui aurait une intégration native avec les fichiers de configuration. Cependant d'autres IDE peuvent surement etre utilisés.

\subsubsection{editeur Atom}

Atom est un éditeur de texte hautement modulaire.
Télécharger atom sur \href{https://atom.io/}{le site officiel}

\subsubsection{Plug-in Platformio}

Le plug-in Plaformio pour Atom permet le déploement rapide depuis l'éditeur des binaires sur les chipsets.

\subsection{outils en ligne de commande : pio}

Platformio met à disposition une commande pour compiler automatiquement le code, pio.

\chapter{Coder le Moteino}

\section{Architecture du moteino}

\section{Principe de code}

\section{Utilisation pio}

\end{document}
