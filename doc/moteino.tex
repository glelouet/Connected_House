\documentclass{report}
\usepackage[french]{babel}
\usepackage[utf8]{inputenc}
\usepackage{hyperref}

\title{Utilisation du projet Moteino}

\begin{document}

\chapter{Contexte}

Le Moteino est un ordinateur de très petite taille et peu cher. Il est basé sur l'Arduino, très modulaire. 

Le but de ce projet est de prendre en main l'architecture embarquée du Moteino : produire du code C++ à installer sur un Moteino, mais surtout comprendre comment manipuler correctement cette architecture.
La production de code C++ demande l'utilisation d'outils particuliers. Moteino propose un IDE minimaliste qui ne convient pas aux projets industriels, nous utilisons donc platformio pour automatiser la gestion des librairies, compiler et envoyer le code sur les chipsets.

\chapter{Liens utiles}

\href{https://github.com/glelouet/arduino}{le git du projet}

\href{http://platformio.org/}{Le site de platformio}

\chapter{Installation des outils}

\chapter{Description du code}

\section{Architecture du moteino}

\section{}

\end{document}
